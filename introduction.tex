\chapter{Introduction}

\lettrine{L}{e} Cirad --- où j'ai effectué mon stage --- est un organisme de recherche spécialisé dans l'agronomie des régions tropicales et subtropicales, et l'un de ses objectif principal est le développement durable desdites régions.
Cependant la notion de développement durable vient avec quelques contraintes.
Notamment, la durabilité implique la limitation des pesticides; et le développement induit la nécessité d'une production agricole efficiente, capable de nourrir dix milliards de personnes d'ici 2050.

Ainsi, il est naturel que le sixième fruit le plus produit au monde, à savoir la mangue\footnote{La sixième production fruitière mondiale est en réalité le groupement des mangues, mangoustans et goyaves \citep{fao}}, soit un sujet de recherche.
C'est d'autant plus vrai que la culture des manguiers (\emph{Mangifera indica L.}) n'est pas toujours facile.
En effet, les manguiers présentent de forts asynchronismes phénologiques, que ce soit à l'intérieur d'une même parcelle entre les différents arbres ou à l'intérieur même d'un arbre entre les différentes branches.
Cela entraîne une floraison et une fructification étalée dans le temps, rendant la gestion des vergers plus difficile.
Ce phénomène entraîne aussi, pour les différents organes du manguier, une fenêtre d'exposition prolongée aux ravageurs, ce qui favorise leur prolifération.
Et ils sont légion. On peut citer la cécidomyie des feuilles (\emph{Procontarinia matteina}) qui s'attaque aux feuilles, la mouche des fruits (\emph{Bactrocera frauenfeldi}) qui ravage les fruits ou encore le charançon du noyau (\emph{Sternochetus mangiferae}) qui détruit les noyaux des fruits.
% Synchroniser la floraison --- à travers l'élagage --- réduirait la fenêtre d'exposition aux ravageurs et \emph{a fortiori} diminuerait leur nombre.

Parmi les ravageurs se trouve également la cécidomyie des fleurs (\emph{Procontarinia mangiferae}). Cette dernière pond ses œufs dans les inflorescences, ce qui provoque des dommages potentiellement importants voire la mort des inflorescences. Et qui dit pas d'inflorescences, dit pas de mangues !

% Une partie du cycle de développement de la cécidomyie a lieu dans le sol, on peut envisager différents recouvrements du sol (par exemple, une bache) qui permettraient de stopper le cycle de développement du ravageur.
% Il est aussi envisagé de réduire la fenêtre d'exposition des inflorescences aux cécidomyies en synchronisant la floraison.
% On propose par la suite un modèle pour mieux appréhender les intéractions entre les cécidomyies et les manguiers.
% Ce modèle devrait permettre de tester des modes de conduites des vergers \emph{in silico} afin de voir si l'on peut fortement diminuer la population de cécidomyies sans utiliser d'intrants phytosanitaires.

Pour limiter les dégâts de la cécidomyie des fleurs sans utiliser de pesticides, deux pistes sont envisagées.
La première serait la synchronisation de la floraison, grâce à l'élagage, ce qui réduirait la fenêtre d'exposition aux ravageurs et limiterait par conséquent leur nombre.
%La seconde repose sur le fait que l'une des phases du cycle de développement de la cécidomyie a lieu dans le sol.
La seconde repose sur le fait que les œufs pondus dans les inflorescences se transforment en larves qui vont ensuite s'enfouir dans le sol.
Restreindre l'accès au sol (avec un enherbement haut par exemple,  qui augmente le trajet des larves pour atteindre la terre et favorise la présence de prédateurs) ou l'empêcher (\emph{e.g.} en utilisant une bâche) devrait \emph{a priori} permettre de réduire la présence de ces ravageurs. 
% Ainsi, empêcher une cécidomyie d'entrer dans le sol devrait \emph{a priori} permettre de réduire la présence de ces ravageurs.
Afin de pouvoir vérifier cette hypothèse, une expérimentation sur un verger a été conduite en 2017 ; une parcelle a été divisée en trois pour tester trois modalités de couverture du sol différentes : un enherbement ras, un paillage synthétique et un enherbement haut. De cette expérimentation, des données ont été récoltées. L'objectif est d'utliser les données pour modéliser les intéractions  entre les cécidomyies et les inflorescences.
Une première version du modèle a été réalisée lors du stage de \citet{laurie}, mon stage en est la suite et a pour objectif d'améliorer le modèle existant puis de tester des modes de conduites des vergers \emph{in silico} pour pouvoir répondre à la question :
\begin{quote}
 \textbf{La modalité de couverture du sol et la synchronisation de la floraison permettent-elles de limiter l'infestation du verger par les cécidomyies des fleurs ?}
\end{quote}

[BALANCER LE PLAN]
