\chapter{Discussion} 

\lettrine{O}{n} revient dans ce chapitre sur les résultats obtenus, les hypothèses faites et les méthodes employées.

Commençons par l'aspect biologique.
On peut faire quelques commentaires sur certaines hypothèses qui ont pu être faites.
En premier lieu, le fait que la venue de femelles exogènes se fasse proportionnellement aux inflorescences est une hypothèse forte.
Ce n'est pas très grave dans le cas des solutions--types qui privilégient les individus endogènes, où les femelles exogènes sont peu nombreuses.
Lorsque les femelles exogènes contribuent fortement aux dynamiques de larves, il pourrait être pertinent de mener une expérimentation afin de confirmer ou d'infirmer cette hypothèse.

Concernant l'hypothèse d'attractivité des inflorescences, ne prendre en compte que les stades phénologiques C, D et E est peut-être un peu radical.
En effet, si le début du stade F commence avec l'apparition de la première fleur cela ne veut pas dire que l'inflorescence devient subitement inattractive pour les cécidomyies.
Si cette attractivité existe bien en réalité, alors ce serait probablement un phénomène plus progressif et le début de la période d'inattractivité interviendrait au milieu du stade F.
Là encore, une expérimentation pourrait s'avérer utile pour déterminer à quel stade phénologique les inflorescences sont le plus attaquées.

La mise en place d'un paramètre de saisonnalité telle que nous l'avons faite revient, en pratique, à éradiquer toutes les femelles (ou les empêcher de pondre) du jour au lendemain.
Cette prédiction du modèle est pour l'instant difficile à expliquer. 
On remarque toutefois sur nos observations (pour chacune des sous-parcelles des deux vergers) une chute brutale du nombre de larves en fin de saison.
Ce qui laisse penser qu'il se passe effectivement quelque chose en fin de saison.
Or, aucune connaissance présente dans la littérature n'explique l'absence de reproduction des cécidomyies autrement que par un changement conséquent de température ou une absence de ressources.
Des expérimentations supplémentaires seront nécessaires pour identifier ce phénomène.
On peut cependant émettre des hypothèses comme, par exemple, une compétition entre la cécidomyie et un autre ravageur qui s'en prendraient aux fleurs des inflorescences et apparaîtraient ainsi qu'à partir du stade phénologique F.

Concernant le fait que les paramètres trouvés ne s'applique pas parfaitement aux dynamiques sur le verger n\textdegree2, cela peut s'expliquer par des conditions différentes entre les deux vergers.
Le verger n\textdegree2 est en pente, et les modalités de couverture du sol n'ont pas la même disposition.
Notamment, la sous-parcelle PS se trouve sur le bord à proximité d'un autre verger, ce qui pourrait expliquer une plus forte présence d'individus exogènes.

On notera également qu'une expérimentation est en cours visant à quantifier le nombre d'individus qui sortent de diapause, ce qui permettra également de confirmer ou d'infirmer l'ordre de grandeur de la valeur trouvée.

% Il faut aussi appuyer sur le fait que la floraison du manguier présente une grande variabilité.
% Nous avons pu observer des dynamiques d'inflorescences très différentes au sein d'une même sous-parcelle suite à un échantillonage différent (voir figure~\ref{fig:inflos}).
% Il est alors difficile de juger la pertinence du modèle en se basant sur si peu d'observations.
% En effet, il suffit que les dynamiques sur lesquelles on calibre le modèle soient atypiques pour que le modèle ne puisse pas se généraliser à d'autres cas.
% Difficile de conclure.

On peut également émettre quelques remarques concernant la méthodologie employée.
Concernant l'algorithme d'optimisation, seul NSGA-II a été utilisé.
Son utilisation a été concluante pour déterminer des solutions optimales des modèles, c'est pourquoi l'on n'en a pas essayé d'autre.
On aurait probablement pu trouver un autre algorithme multicritères qui converge plus vite ou plus près du front de Pareto.
Cependant la diversité des solutions proposées par NSGA-II est déjà grande.
Et avoir des paramètres plus précis n'est peut-être pas pertinent dans la mesure où le phénomène modélisé est très variable et que l'on calibre le modèle sur des données qui ne sont pas elles-mêmes extrêmement précises.

On peut aussi remettre en question la classification effectuée.
Et surtout le choix d'effectuer la classification sur les valeurs de paramètres.
On part ici du principe que deux jeux de valeurs de paramètres dont les valeurs sont proches produiront des dynamiques similaires.
Ce principe ne s'applique pas tous le temps, notamment dans les systèmes chaotiques, où des valeurs initiales très proches peuvent aboutir à des résultats très différents.
Peut-être que dans notre modèle, il se produit --- dans une certaine mesure --- un phénomène similaire, avec notamment la possibilité d'un «effet seuil» chez certains paramètres, qui pourrait produire des dynamiques très différentes malgré des valeurs de paramètres proches.
