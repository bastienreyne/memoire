\chapter{Expression de la pupaison en fonction de la température}
\label{chap:pupaison}

Jusqu'à présent nous avions dans notre modèle $p_{\text{p}}$ fixé à 0.77. 
En utilisant une constante, on ne prend pas du tout en compte les conditions externes qui pourrait influencer sur le cycle de développement des cécidomyies (\emph{e.g.} les conditions météorologiques).
 
Le but est ici d'obtenir une valeur de $p_{\text{pup}}$ qui dépendrait de la température. On choisit la température car cette variable est accesible et qu'elle a, \emph{a minima}, un effet sur la durée de la diapause \citep{pauldiap}.

On récupère les données de l'article de \citet{pauldiap} pour essayer de voir s'il y a un lien entre température et durée de pupaison.

À chaque date --- pour lesquelles on a des données sur le nombre de larves et le nombre de larves en pupaison --- on calcule la proportion de larves en pupaison.
On extrait ensuite les températures moyennes sur 15 jours (de 7 jours avant jusqu'à 7 jours après) pour chaque jour où l'on a des données disponibles.
Le choix de prendre la température moyenne sur 15 jours est fait pour prendre au mieux en compte les conditions climatiques qu'il y a eu avant l'enfouissement (et notamment au moment de la ponte) et après l'enfouissement (durant la pépriode de pupaison).
On effectue une régression linéaire simple de la proportion de larves en pupaison par la température.

On fixera le seuil du risque de première espèce à $\alpha = 5\%$ pour le test de non-nullité des coefficients.
Les résultats sont visibles dans la table~\ref{tab:lm2}.
On en conclut que les coefficients associés à l'ordonnée à l'origine et à la température ne sont pas nuls et que l'on peut écrire
\[
p_{\text{p}} = 1.9555 - 0.055\times t_{15j}.
\]

\begin{table}[hb]
\centering
\caption{Régression linéaire simple de la proportion d'individus en pupaison par la température moyenne sur 15 jours}
\label{tab:lm2}
\begin{tabular}{rrrrr}
 & Estimate & Std. Error & t value & Pr($>$$|$t$|$) \\ 
  \hline
(Intercept) & 1.9555 & 0.3665 & 5.34 & 0.0002 \\ 
  temp15j & -0.0550 & 0.0160 & -3.43 & 0.0050 \\ 
\end{tabular}
\end{table}
