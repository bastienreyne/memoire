\chapter{Conclusion} 

\lettrine{S}{i} l'on considère que l'objectif principal du stage était de savoir quelle modalité de couverture du sol et quelle dynamique d'inflorescences permettent de limiter l'impact de la cécidomyie des fleurs sur un verger de manguiers, alors le stage est un échec.
Il faut néanmoins relativiser. Il y avait plusieurs objectifs au stage.

Le premier était de modéliser le système manguier -- cécidomyies des fleurs.
Le modèle existe et il est fonctionnel.
Cet objectif est atteint.

Le deuxième objectif était d'analyser le fonctionnement dudit système.
Bien que nous n'avons pas été capable de proposer une solution reproduisant les observations effectuées sur le terrain, cet objectif est partiellement rempli.
En effet, le modèle semble mettre en évidence l'existence d'un phénomène en fin de saison responsable de la diminution du nombre de cécidomyies, ce qui permet 
d'envisager de nouvelles pistes pour mieux comprendre le système.

Le troisième objectif était de réaliser des tests \emph{in silico} afin d'évaluer le meilleur mode de gestion des vergers.
Cet objectif n'a en revanche pas été atteint.

À l'issue de ce stage, on peut envisager plusieurs pistes pour de futures tentatives.
Par exemple, réaliser de nouvelles expérimentations sur le terrain pour acquérir plus de données.
Et aussi s'intéresser à la possible compétition entre la cécidomyie des fleurs et d'autres ravageurs.
Il faudra \emph{a priori} avoir plus de connaissances et/ou de données avant d'entreprendre à nouveau une démarche de modélisation.

\paragraph{ } D'un point de vue personnel, ce stage aura été une réussite.
D'abord d'un point de vue technique. J'ai pu appliquer des concepts mathématiques appris en cours à un cas concret.
J'ai aussi été amené à apprendre par moi-même non vus en cours, et à les appliquer. Cela fut très intéressant.

Ensuite d'un point de vue professionnel, j'ai pu avoir une première expérience dans le domaine de la recherche.
Expérience fort plaisante, qui m'a permis de me faire une idée plus précise du métier de chercheur.
Et de rappeler certaines évidences : les résultats obtenus ne sont pas toujours présents ou ne sont pas toujours ceux que l'on cherche initialement, et que cette incertitude est au cœur même du métier.
