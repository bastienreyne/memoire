\chapter{Conclusion} 

\lettrine{S}{i} l'on considère que l'objectif principal du stage était d'améliorer la compréhension du système manguier -- cécidomyies, alors le stage est une réussite.
Plus précisément, il y avait trois objectifs.

Le premier était de modéliser le système manguier -- cécidomyies des fleurs.
Le modèle existe et il est fonctionnel.
Cet objectif est atteint.

Le deuxième objectif était d'analyser le fonctionnement dudit système.
Les grandes composantes du système ont été identifiés et modélisées.
On n'a cependant validé que partiellement les paramètres sur le verger n\textdegree2.
Il est important de noter que le modèle fait une prédiction et semble mettre en évidence l'existence d'un phénomène en fin de saison, jusqu'alors non identifié, responsable de la diminution du nombre de cécidomyies --- ce qui permet d'envisager de nouvelles pistes pour mieux comprendre le système.

Le troisième objectif était de réaliser des tests \emph{in silico} afin d'évaluer le meilleur mode de gestion des vergers.
Cet objectif était conditionné aux deux autres.
Bien que la finalisation du modèle a pris du temps, la dernière version du modèle doit permettre de faire une telle analyse dans le futur.
% Et le deuxième objectif a mis en évidence qu'il nous manquait des connaissances dans la compréhension du système, rendant les simulations \emph{in silico} peu pertinentes.

À l'issue de ce stage, on peut envisager plusieurs pistes pour de futures améliorations.
Par exemple, réaliser de nouvelles expérimentations sur le terrain pour acquérir plus de données sur l'attractivité des différents stades phénologiques des inflorescences.
Et aussi s'intéresser à la possible compétition entre la cécidomyie des fleurs et d'autres ravageurs.
Il faudra \emph{a priori} acquérir plus de connaissances et/ou de données avant d'entreprendre à nouveau une démarche de modélisation.

\paragraph{ } D'un point de vue personnel, ce stage aura été une réussite.
D'abord d'un point de vue technique. J'ai pu appliquer des concepts mathématiques appris en cours à un cas concret.
J'ai aussi été amené à apprendre par moi-même des concepts non vus en cours relatifs à l'optimisation et l'analyse de sensibilité d'un modèle, et à les appliquer. Cela fut très intéressant.

Ensuite d'un point de vue professionnel, j'ai pu avoir une première expérience dans le domaine de la recherche.
Expérience fort plaisante, qui m'a permis de me faire une idée plus précise du métier de chercheur.
Et de rappeler certaines évidences : les résultats obtenus ne sont pas toujours ceux que l'on cherche initialement, et que cette incertitude est au cœur même du métier.
Et c'est ce qui rend ce métier passionnant.
